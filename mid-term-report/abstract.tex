%\clearpage
%\thispagestyle{empty}
%~\clearpage
\begin{center}
{\Large \bf ABSTRACT}
\end{center}
\noindent
Air pollution cause massive damages to life with several pulmonary ailments. Thus,
it is important to monitor air pollution levels in the atmosphere. Air pullution in India has become a serious issue and it has brought down life expectancy by 2.6 years[1].      In this project,
students will be working on a data collected using IoT sensors in the Mandi district
for predicting air pollution levels of the air pollutants PM$_{2.5}$, PM$_{10}$ and the gaseous air pollutants NO$_{x}$, SO$_{2}$, CO and O$_{3}$  using
statistical, Machine Learning and Deep Learning based time-series forecasting methods. The  aim of the project is to produce such time series forecasting models which can predict the level of air pollutants given the past concentrations and the climatic indicators.
\\{\bf Keywords}:
{\it Add only IEEE keyword.}
\newpage
\clearpage 
